Scientific models in the climate domain are predominantly developed in Fortran for its high-performance and science-oriented nature. On the other hand, much of Machine Learning (ML) research is conducted in Python owing to its rich ecosystem of ML frameworks, e.g. PyTorch. 

\textbf{FTorch}~\citep{Atkinson2025} is an open source \href{https://github.com/Cambridge-ICCS/FTorch}{\textcolor{black}{\faGithub}}~library developed by \emph{ICCS} allowing the user to do hybrid modeling by interfacing PyTorch ML models with Fortran code. \\
\textbf{Features}:
\begin{itemize}
    \item Binds Fortran to PyTorch's C\texttt{++} backend (no \faPython~runtime) via~\texttt{iso\_c\_binding} module
    \item Provides a user-friendly Fortran API close to PyTorch API
    \item Enables \emph{zero-copy} data transfer between languages
    \item Handles Fortran \emph{column-major} vs. C/C\texttt{++} \emph{row-major} memory layout differences
    \item Supports CPU and GPU (CUDA, XPU, MPS backends)
    \item Tested on UNIX and Windows operating systems
\end{itemize}

